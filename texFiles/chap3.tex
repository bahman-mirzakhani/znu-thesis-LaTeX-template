% !TeX root = ../thesis.tex
\chapter{مواد و روش‌ها}
\section{شکل}
\begin{figure}[!htb]
	\centering
	\includegraphics[width=.3\linewidth]{example-image}
	\caption{توضیحات شکل}
	\label{fig:test}
\end{figure}

\subsection{چند شکل در کنار هم}
\begin{figure}[!htb]
	\centering
	\begin{subfigure}{.2\linewidth}
		\centering
		\includegraphics[width=\linewidth]{example-image-a}
		\caption{}
		\label{subfig:exam1}
	\end{subfigure}
	\hspace{0.5cm}
	\begin{subfigure}{.2\linewidth}
		\centering
		\includegraphics[width=\linewidth]{example-image-b}
		\caption{}
		\label{subfig:exam2}
	\end{subfigure}\\
	\begin{subfigure}{.2\linewidth}
		\centering
		\includegraphics[width=\linewidth]{example-image-c}
		\caption{}
		\label{subfig:exam3}
	\end{subfigure}
	\caption[توضیحات کلی]{توضیحات کلی.
		\subref{subfig:exam1})
		توضیحات اول،
		\subref{subfig:exam2})
		توضیحات دوم،
		\subref{subfig:exam3})
		توضیحات سوم}
	\label{fig:subfig}
\end{figure}

\section{جدول}
\begin{table}[!htb]
	\centering
	\caption{نمونه جدول}
	\label{tab:test}
	\begin{tabular}{ccc}
		\toprule
		نام & نام‌خانوادگی & کدملی
		\\\midrule
		اول & دوم & 1234567890
		\\
		سوم & چهارم & 0987654321
		\\
		پنجم & ششم & 5432106789
		\\\bottomrule
	\end{tabular}
\end{table}

\subsection{ادغام ستون‌ها}
\begin{table}[!htb]
	\centering
	\caption{ادغام ستون‌ها}
	\label{tab:mergeCol}
	\begin{tabular}{|c|c|c|}
		\hline
		\multicolumn{2}{|c|}{\RL{نام و نام‌خانوادگی}}
		& کدملی
		\\\hline
		اول & دوم & 1234567890
		\\
		سوم & چهارم & 0987654321
		\\
		پنجم & ششم & 5432106789
		\\\hline
	\end{tabular}
\end{table}

\subsection{ادغام سطرها}
\begin{table}[!htb]
	\centering
	\caption{ادغام سطرها}
	\label{tab:newTest}
	\begin{tabular}{|c|c|c|}
		\hline
		نام & نام‌خانوادگی & کدملی
		\\\hline
		اول & دوم & 1234567890
		\\\hline
		سوم & چهارم &
		\multirow{2}{*}{0987654321}
		\\\cline{1-2}
		پنجم & ششم & 
		\\\hline
	\end{tabular}
\end{table}

\section{روابط ریاضی}
برای آشنایی با انواع محیط‌های ریاضی و کاربردشان، از راهنمای بستهٔ
\texttt{amsmath}
استفاده کنید. کافیست در پنجرهٔ
\lr{Command Prompt}
(یا همان
\lr{cmd})،
فرمانِ
\lr{texdoc amsmath}
صادر گردد.

\begin{description}
	\item[توجه:]
	توابع و برخی حروف ریاضی مانند عدد نپر، عملگر دیفرانسیل و یکّهٔ موهومی و \ldots،
	بایستی ایستاده حروف‌چینی شوند. برای مثال:
	\[
	\E^{\I \theta} = \cos \theta + \I \sin \theta, \quad
	F(x) = \int f(x) \diff x
	\]
\end{description}