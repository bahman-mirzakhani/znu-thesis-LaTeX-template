% !TeX root = ../thesis.tex
% www.parsilatex.com/wiki/راهنمای_ایجاد_واژه%E2%80%8Cنامه
%====================================================
% تعیین استایل برای واژه‌نامهٔ فارسی به انگلیسی:
\newglossarystyle{myFaToEn}{%
	\renewenvironment{theglossary}{}{}%
%	\renewcommand*{\glsgroupskip}{\vskip 5mm}%
	\renewcommand*{\glsgroupheading}[1]{%
		\subsection*{\glsgetgrouptitle{##1}}%
	}%
	\renewcommand*{\glossentry}[2]{%
		\noindent\glsentryname{##1}\space\dotfill\space\glsentrytext{##1}\par		
	}%
}
%====================================================
% تعیین استایل برای واژه‌نامهٔ انگلیسی به فارسی:
\newglossarystyle{myEntoFa}{%
	\renewenvironment{theglossary}{}{}%
%	\renewcommand*{\glsgroupskip}{\vskip 5mm}%
	\renewcommand*{\glsgroupheading}[1]{%
		\begin{LTR}
			\subsection*{\glsgetgrouptitle{\lr{##1}}}
		\end{LTR}%
	}%
	\renewcommand*{\glossentry}[2]{%
		\noindent\glsentrytext{##1}\space\dotfill\space\glsentryname{##1}\par		
	}%
}
%====================================================
% تعیین استایل برای فهرست اختصارات:
\renewcommand*{\acronymname}{فهرست اختصارات}
\newglossarystyle{myAbbrlist}{%
	\renewenvironment{theglossary}{}{}%
	\renewcommand*{\glsgroupskip}{\vskip 5mm}%
	\renewcommand*{\glsgroupheading}[1]{%
		\begin{LTR}
			\subsection*{\glsgetgrouptitle{\lr{##1}}}
		\end{LTR}%
	}%
%====================================================
% نحوهٔ نمایش اختصارات؛ کوچک، سمت چپ و بزرگ، سمت راست:
	\renewcommand*{\glossentry}[2]{%
		\noindent\lr{\glsentrytext{##1}\space\dotfill\space\Glsentrylong{##1}}\par
	}%
}
%====================================================
% در اینجا عباراتی مثل glg، gls، glo و… پسوند فایل‌هایی است که برای xindy به کار می‌روند. 
\newcommand*{\enFaGlossName}{واژه‌نامهٔ انگلیسی به فارسی}
\newcommand*{\faEnGlossName}{واژه‌نامهٔ فارسی به انگلیسی}
\newglossary[glg]{english}{gls}{glo}{\enFaGlossName}
\newglossary[blg]{persian}{bls}{blo}{\faEnGlossName}
\makeglossaries
\glsdisablehyper
%====================================================
% تعاریف مربوط به تولید واژه‌نامه و فهرست اختصارات:
\NewCommandCopy{\oldgls}{\gls}
\NewCommandCopy{\oldglspl}{\glspl}

\makeatletter
\renewrobustcmd*{\gls}{\@ifstar{\@msgls}{\@mgls}}
\newcommand*{\@mgls}[1]{%
	\ifthenelse % {test}{then clause}{else clause}
	{\equal{\glsentrytype{#1}}{english}}%
	{\oldgls{#1}\glsuseri{f-#1}}%
	{\lr{\oldgls{#1}}}%
}
\newcommand*{\@msgls}[1]{%
	\ifthenelse % {test}{then clause}{else clause}
	{\equal{\glsentrytype{#1}}{english}}%
	{\glstext{#1}\glsuseri{f-#1}}%
	{\lr{\glsentryname{#1}}}%
}
\renewrobustcmd*{\glspl}{\@ifstar{\@msglspl}{\@mglspl}}
\newcommand*{\@mglspl}[1]{%
	\ifthenelse % {test}{then clause}{else clause}
	{\equal{\glsentrytype{#1}}{english}}%
	{\oldglspl{#1}\glsuseri{f-#1}}%
	{\oldglspl{#1}}%
}
\newcommand*{\@msglspl}[1]{%
	\ifthenelse % {test}{then clause}{else clause}
	{\equal{\glsentrytype{#1}}{english}}%
	{\glsplural{#1}\glsuseri{f-#1}}%
	{\glsentryplural{#1}}%
}
\makeatother

\newcommand{\newword}[4]{%
	\newglossaryentry{#1}{%
		type={english},%
		name={\lr{#2}},%
		plural={#4},%
		text={#3},%
		description={}%
	}%
	\newglossaryentry{f-#1}{%
		type={persian},%
		name={#3},%
		text={\lr{#2}},%
		description={}%
	}%
}
%====================================================
% طبق این دستور، در اولین باری که واژهٔ مورد نظر از واژه‌نامه وارد شود، پاورقی زده می‌شود. 
\defglsentryfmt[english]{%
	\glsgenentryfmt\ifglsused{\glslabel}{}{%
		\LTRfootnote{\glsentryname{\glslabel}}%
	}%
}
%====================================================
% طبق این دستور، در اولین باری که واژهٔ مورد نظر از فهرست اختصارات وارد شود، پاورقی زده می‌شود. 
\defglsentryfmt[acronym]{%
	\ifglsused{\glslabel}{}{%
		\rl{\LTRfootnote{\glsentrydesc{\glslabel}}}%
	}%
	\glsentryname{\glslabel}%
}
%====================================================
% دستور برای قراردادن فهرست اختصارات:
\newcommand{\printabbreviation}{%
	{\clearpage
		\setstretch{1}%
		\setglossarystyle{myAbbrlist}%
		\Oldprintglossary[type=acronym]%
	}%
}
%====================================================
% دستور برای قراردادن واژه‌نامه‌ها:
\NewCommandCopy{\Oldprintglossary}{\printglossary}
\renewcommand{\printglossary}{%
	\twocolumn
	{\setstretch{1}%
		\setglossarystyle{myFaToEn}%
		\Oldprintglossary[type=persian]%
		\setglossarystyle{myEntoFa}%
		\Oldprintglossary[type=english]%
	}%
	\onecolumn
}%
%====================================================
