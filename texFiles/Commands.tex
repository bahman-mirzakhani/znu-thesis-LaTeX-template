% !TeX root = ../thesis.tex
% ===========================================================
% بسته‌های ریاضی:
\usepackage{mathtools} % amsmath
\usepackage{amsfonts}
\usepackage{amssymb}
\usepackage{amsthm}
\usepackage{mleftright}
\RenewCommandCopy{\left}{\mleft}
\RenewCommandCopy{\right}{\mright}
% ===========================================================
% عکس و مسیر عکس‌ها:
\usepackage{graphicx}
\graphicspath{{./pictures/}}
\usepackage{subcaption}
% ===========================================================
% بسته‌ای برای واردکردن pdf:
\usepackage{pdfpages}
% ===========================================================
% بسته‌ای برای تنظیمات کپشن‌ها:
\usepackage{caption}
\DeclareCaptionFont{mytwelve}{\fontsize{10.5}{12.5}\selectfont}
\captionsetup{font=mytwelve, aboveskip=3pt, width=0.85\linewidth}
\captionsetup[algorithm]{labelsep=colon, labelfont=md}
% ===========================================================
% تنظیم ارتفاع و فونت جدول‌ها:
\usepackage{array}
\makeatletter
\newcommand*\my@starttable[1][]{%
	\@float{table}[#1]\fontsize{11}{13.5}\selectfont
	\renewcommand{\arraystretch}{1.75}%
	\addtolength{\extrarowheight}{-2pt}%
}
\patchcmd{\table}{\@float{table}}{\my@starttable}{\PackageInfo{mysty}{Table environment patched successfully.}}{\PackageWarning{mysty}{Could not patch table environment.}}
\makeatother
% بسته‌های مرتبط با جدول:
\usepackage{booktabs}
\usepackage{multirow}
% ===========================================================
% بسته‌ای برای محیط کدها و رنگ‌ها:
\usepackage{listings}
\usepackage{xcolor}
% بسته‌ای برای مدیریت ارجاع به منابع:
\ifznuUseRefNum
	\usepackage{cite}
	\bibliographystyle{unsrt-fa}
\else
	\usepackage{natbib}
	\bibliographystyle{chicago-fa}
\fi
% بسته‌ای برای جلوگیری از بیوه‌شدن!:
\usepackage[defaultlines=2, all]{nowidow}
% ===========================================================
% بسته‌های لازم برای نوشتن الگوریتم:
\usepackage{algorithm}
\usepackage{algorithmic}
\numberwithin{algorithm}{chapter}
% ===========================================================
% بسته‌های فراخوانی‌شده:
% alphalph, atbegshi, etoolbox, fancyhdr, geometry
% multicol, newfloat, setspace, titlesec, tocbibind, tocloft
% ===========================================================
% بسته‌های جدید و مورد نیاز خود را اینجا فرابخوانید:

% ===========================================================
% بسته‌ای برای ایجاد نمایه (در صورت وجود):
\ifznuUseIndex
	\usepackage{makeidx}
	\makeindex
\fi
% ===========================================================
% هم‌راستاکردن منابعی که url دارند:
\usepackage[hyphens]{url}
\Urlmuskip=0mu plus 1mu
\urlstyle{tt}
% ===========================================================
% امکان جهش به لینک‌ها:
\makeatletter
\ifznuFinal
	\usepackage[hidelinks]{hyperref}
\else
	\usepackage[pagebackref]{hyperref}
\fi
\pretocmd{\titleInfoFa}{%
	\hypersetup{%
		pdfauthor=\@fullNameFa,%
		pdftitle=\@titleFa,%
		pdfsubject=\ifznuBscThesis پایان‌نامهٔ کارشناسی\else
		\ifznuMscThesis پایان‌نامهٔ کارشناسی‌ارشد\else رسالهٔ دکتری\fi\fi
	}
}{}{}
\makeatother
% ===========================================================
% بسته‌ای برای فهرست اختصارات و واژه‌نامه:
\ifznuUseGlossaries
	\usepackage[%
		xindy,%
		toc,%
		acronym%
	]{glossaries}
	\makeatletter
	\@ifpackagelater{glossaries}{2022/10/14}%
		{\setacronymstyle{footnote}}{}
	\makeatother
\fi
% ===========================================================
% برای نوشتن فایل‌های بیشتر:
\usepackage{morewrites}
% ===========================================================
% بستهٔ زی‌پرشین و فونت‌ها:
\usepackage[%
	perpagefootnote=on%
]{xepersian}
\settextfont[%
	Path={./fonts/},%
	Scale=1.25,%
	ItalicFont={IRLotusICEE_Iranic.ttf},%
	BoldFont={IRLotusICEE_Bold.ttf},%
	BoldItalicFont={IRLotusICEE_BoldIranic.ttf},%
]{IRLotusICEE.ttf}
\ifznuDigitFa
	\setmathdigitfont[%
		Path={./fonts/},%
		Scale=1.25,%
		ItalicFont={IRLotusICEE_Iranic.ttf},%
		BoldFont={IRLotusICEE_Bold.ttf},%
		BoldItalicFont={IRLotusICEE_BoldIranic.ttf},%
	]{IRLotusICEE.ttf}
\fi
\defpersianfont\titr[Path={./fonts/}, Scale=1.2]{XB Titre.ttf}
\defpersianfont\nastaliq[Path={./fonts/}, Scale=2]{IranNastaliq.ttf}
\defpersianfont\IRZar[Path={./fonts/}, Scale=1.1]{IRZarBold.ttf}
\SepMark{-}
% ===========================================================
% دستورات لازم برای تعریف محیط‌های شبه‌قضیه:
\theoremstyle{plain}
\newtheorem{thm}{قضیه}[chapter]
\newtheorem{lem}[thm]{لم}
\newtheorem{prop}[thm]{گزاره}
\newtheorem*{cor}{نتیجه}

\theoremstyle{definition}
\newtheorem{defn}[thm]{تعریف}
\newtheorem{exmp}[thm]{مثال}

\theoremstyle{remark}
\newtheorem*{rem}{تبصره}

\AtBeginDocument{\renewcommand{\proofname}{برهان}}
% ===========================================================
% دستورهای لازم برای تعریف ترجمهٔ دستورات الگوریتم:
% برگرفته از قالب دانشگاه تهران با کمی تغییرات
\makeatletter
\renewcommand{\algorithmicrequire}{\if@RTL\textbf{ورودی:}\else\textbf{Require:}\fi}
\renewcommand{\algorithmicensure}{\if@RTL\textbf{خروجی:}\else\textbf{Ensure:}\fi}
\renewcommand{\algorithmicend}{\if@RTL\textbf{پایانِ}\else\textbf{end}\fi}
\renewcommand{\algorithmicif}{\if@RTL\textbf{اگر}\else\textbf{if}\fi}
\renewcommand{\algorithmicthen}{\if@RTL\textbf{آنگاه}\else\textbf{then}\fi}
\renewcommand{\algorithmicelse}{\if@RTL\textbf{وگرنه}\else\textbf{else}\fi}
\renewcommand{\algorithmicfor}{\if@RTL\textbf{برای}\else\textbf{for}\fi}
\renewcommand{\algorithmicforall}{\if@RTL\textbf{برای هر}\else\textbf{for all}\fi}
\renewcommand{\algorithmicdo}{\if@RTL\textbf{انجام بده}\else\textbf{do}\fi}
\renewcommand{\algorithmicwhile}{\if@RTL\textbf{تا زمانی که}\else\textbf{while}\fi}
\renewcommand{\algorithmicloop}{\if@RTL\textbf{تکرار کن}\else\textbf{loop}\fi}
\renewcommand{\algorithmicrepeat}{\if@RTL\textbf{تکرار کن}\else\textbf{repeat}\fi}
\renewcommand{\algorithmicuntil}{\if@RTL\textbf{تا زمانی که}\else\textbf{until}\fi}
\renewcommand{\algorithmicprint}{\if@RTL\textbf{چاپ کن}\else\textbf{print}\fi}
\renewcommand{\algorithmicreturn}{\if@RTL\textbf{بازگردان}\else\textbf{return}\fi}
\renewcommand{\algorithmicand}{\if@RTL\textbf{و}\else\textbf{and}\fi}
\renewcommand{\algorithmicor}{\if@RTL\textbf{و یا}\else\textbf{or}\fi}
\renewcommand{\algorithmicxor}{\if@RTL\textbf{یا}\else\textbf{xor}\fi}
\renewcommand{\algorithmicnot}{\if@RTL\textbf{نقیض}\else\textbf{not}\fi}
\renewcommand{\algorithmicto}{\if@RTL\textbf{تا}\else\textbf{to}\fi}
\renewcommand{\algorithmicinputs}{\if@RTL\textbf{ورودی‌ها}\else\textbf{inputs}\fi}
\renewcommand{\algorithmicoutputs}{\if@RTL\textbf{خروجی‌ها}\else\textbf{outputs}\fi}
\renewcommand{\algorithmicglobals}{\if@RTL\textbf{متغیّرهای عمومی}\else\textbf{globals}\fi}
\renewcommand{\algorithmicbody}{\if@RTL\textbf{انجام بده}\else\textbf{do}\fi}
\renewcommand{\algorithmictrue}{\if@RTL\textbf{درست}\else\textbf{true}\fi}
\renewcommand{\algorithmicfalse}{\if@RTL\textbf{نادرست}\else\textbf{false}\fi}
\newcommand{\GOTO}[1]{\if@RTL\textbf{برو به}\else\textbf{go to}\fi~\ref{#1}}
\makeatother
% ===========================================================
% تغییر نوع شماره‌گذاری الگوریتم‌ها:
\makeatletter
\renewcommand{\thealgorithm}{\thechapter\@SepMark\arabic{algorithm}}
\makeatother
% ===========================================================
% دستورات لازم برای اختصارات و واژه‌نامه‌ها:
\ifznuUseGlossaries
	% !TeX root = ../thesis.tex
% www.parsilatex.com/wiki/راهنمای_ایجاد_واژه%E2%80%8Cنامه
%====================================================
% تعیین استایل برای واژه‌نامهٔ فارسی به انگلیسی:
\newglossarystyle{myFaToEn}{%
	\renewenvironment{theglossary}{}{}%
%	\renewcommand*{\glsgroupskip}{\vskip 5mm}%
	\renewcommand*{\glsgroupheading}[1]{%
		\subsection*{\glsgetgrouptitle{##1}}%
	}%
	\renewcommand*{\glossentry}[2]{%
		\noindent\glsentryname{##1}\space\dotfill\space\glsentrytext{##1}\par		
	}%
}
%====================================================
% تعیین استایل برای واژه‌نامهٔ انگلیسی به فارسی:
\newglossarystyle{myEntoFa}{%
	\renewenvironment{theglossary}{}{}%
%	\renewcommand*{\glsgroupskip}{\vskip 5mm}%
	\renewcommand*{\glsgroupheading}[1]{%
		\begin{LTR}
			\subsection*{\glsgetgrouptitle{\lr{##1}}}
		\end{LTR}%
	}%
	\renewcommand*{\glossentry}[2]{%
		\noindent\glsentrytext{##1}\space\dotfill\space\glsentryname{##1}\par		
	}%
}
%====================================================
% تعیین استایل برای فهرست اختصارات:
\renewcommand*{\acronymname}{فهرست اختصارات}
\newglossarystyle{myAbbrlist}{%
	\renewenvironment{theglossary}{}{}%
	\renewcommand*{\glsgroupskip}{\vskip 5mm}%
	\renewcommand*{\glsgroupheading}[1]{%
		\begin{LTR}
			\subsection*{\glsgetgrouptitle{\lr{##1}}}
		\end{LTR}%
	}%
%====================================================
% نحوهٔ نمایش اختصارات؛ کوچک، سمت چپ و بزرگ، سمت راست:
	\renewcommand*{\glossentry}[2]{%
		\latin\noindent\glsentrytext{##1}\space\dotfill\space\Glsentrylong{##1}\par
	}%
}
%====================================================
% در اینجا عباراتی مثل glg، gls، glo و… پسوند فایل‌هایی است که برای xindy به کار می‌روند. 
\newcommand*{\enFaGlossName}{واژه‌نامهٔ انگلیسی به فارسی}
\newcommand*{\faEnGlossName}{واژه‌نامهٔ فارسی به انگلیسی}
\newglossary[glg]{english}{gls}{glo}{\enFaGlossName}
\newglossary[blg]{persian}{bls}{blo}{\faEnGlossName}
\makeglossaries
\glsdisablehyper
%====================================================
% تعاریف مربوط به تولید واژه‌نامه و فهرست اختصارات:
\NewCommandCopy{\oldgls}{\gls}
\NewCommandCopy{\oldglspl}{\glspl}

\makeatletter
\renewrobustcmd*{\gls}{\@ifstar{\@msgls}{\@mgls}}
\newcommand*{\@mgls}[1]{%
	\ifthenelse % {test}{then clause}{else clause}
	{\equal{\glsentrytype{#1}}{english}}%
	{\oldgls{#1}\glsuseri{f-#1}}%
	{\lr{\oldgls{#1}}}%
}
\newcommand*{\@msgls}[1]{%
	\ifthenelse % {test}{then clause}{else clause}
	{\equal{\glsentrytype{#1}}{english}}%
	{\glstext{#1}\glsuseri{f-#1}}%
	{\lr{\glsentryname{#1}}}%
}
\renewrobustcmd*{\glspl}{\@ifstar{\@msglspl}{\@mglspl}}
\newcommand*{\@mglspl}[1]{%
	\ifthenelse % {test}{then clause}{else clause}
	{\equal{\glsentrytype{#1}}{english}}%
	{\oldglspl{#1}\glsuseri{f-#1}}%
	{\oldglspl{#1}}%
}
\newcommand*{\@msglspl}[1]{%
	\ifthenelse % {test}{then clause}{else clause}
	{\equal{\glsentrytype{#1}}{english}}%
	{\glsplural{#1}\glsuseri{f-#1}}%
	{\glsentryplural{#1}}%
}
\makeatother

\newcommand{\newword}[4]{%
	\newglossaryentry{#1}{%
		type={english},%
		name={\lr{#2}},%
		plural={#4},%
		text={#3},%
		description={}%
	}%
	\newglossaryentry{f-#1}{%
		type={persian},%
		name={#3},%
		text={\lr{#2}},%
		description={}%
	}%
}
%====================================================
% طبق این دستور، در اولین باری که واژهٔ مورد نظر از واژه‌نامه وارد شود، پاورقی زده می‌شود. 
\defglsentryfmt[english]{%
	\glsgenentryfmt\ifglsused{\glslabel}{}{%
		\LTRfootnote{\glsentryname{\glslabel}}%
	}%
}
%====================================================
% طبق این دستور، در اولین باری که واژهٔ مورد نظر از فهرست اختصارات وارد شود، پاورقی زده می‌شود. 
\defglsentryfmt[acronym]{%
	\ifglsused{\glslabel}{}{%
		\rl{\LTRfootnote{\glsentrydesc{\glslabel}}}%
	}%
	\glsentryname{\glslabel}%
}
%====================================================
% دستور برای قراردادن فهرست اختصارات:
\newcommand{\printabbreviation}{%
	{\clearpage
		\setstretch{1}%
		\setglossarystyle{myAbbrlist}%
		\Oldprintglossary[type=acronym]%
	}%
}
%====================================================
% دستور برای قراردادن واژه‌نامه‌ها:
\NewCommandCopy{\Oldprintglossary}{\printglossary}
\renewcommand{\printglossary}{%
	\twocolumn
	{\setstretch{1}%
		\setglossarystyle{myFaToEn}%
		\Oldprintglossary[type=persian]%
		\setglossarystyle{myEntoFa}%
		\Oldprintglossary[type=english]%
	}%
	\onecolumn
}%
%====================================================
\fi
% ===========================================================
% تعریف برخی حروف ایستاده در محیط ریاضی:
\newcommand{\diff}{\ensuremath{\,\mathrm d}}
\newcommand{\I}{\ensuremath{\mathrm i}}
\newcommand{\E}{\ensuremath{\mathrm e}}
% ===========================================================